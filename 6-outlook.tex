\section{Discussion and Outlook}

We propose a method for creating a cross-dialectal lexical resource for Low German using LLOD technologies. This approach is particularly suited to a language that lacks a standardized written form, exhibits multiple conflicting orthographies, and shows significant internal variation in phonology, spelling, and grammar. 
We provide a conversion of the \emph{WöWö} dictionary of the Dithmarschen dialect of North Low Saxon into RDF and use this as a lexical backbone. In a second processing step, this was enriched with cross-dialectal links based on formal agreement of \emph{WöWö} lemmas with lexical entries from dictionaries of 5 other Low German dialects.
This data is provided as RDF data, with three files representing the original \emph{WöWö} and one RDF file per external dictionaries. These RDF files define lexical entries and their respective canonical forms, but they do not provide additional details beyond the location of the corresponding lexical entry on the web -- the URI of the lexical entry is the URL of the underlying lemma. With the external dictionaries not providing an RDF view on their content, this is not actually linked data, as these URIs do not resolve to machine-readable data, but it is possible to query the graph and to provide a tabular export that not only includes (excerpts of) \emph{WöWö} information, but also links with external dictionaries.

We provide an HTML view on this tabular export, and for a human, this HTML file (resp., for a machine, the underlying RDF data) is actually capable of serving as a ``digital Rosetta Stone'', linking dictionaries and mapping corresponding words across dialects -- without resorting to a standard variety or spelling (which, for the case of Low German, does not exist). 
% mögliche nutzung: es gibt keine dialektübergreifenden paralleltexte, aber mit verknüpften wörterbüchern könnte man multidialektale word embeddings (und, darauf aufbauend, multidialektale contextualized embeddings) induzieren. jeder der drei dialekte hat eine eigene literatur in unterschiedlichen orthographien.
Aside from supporting speakers and learners in their exploration of interdialectal differences and similarities, this approach also enables new applications in the technical realm: Since there are no cross-dialectal parallel texts for Low German, linking dictionaries could facilitate the induction of multidialectal word embeddings -- and, building upon that, multidialectal contextualized embeddings. Each of the dialects examined here has its own literary tradition, written in different orthographies.

%% die daten können sich gegenseitig anreichern. wir haben definitionen und belege aus mehreren wörterbüchern. auch bei der phonologie: die vollwertige erfassung der phonologie erfordert, ein komplexes inventar von ê- und ô-Phonemen zu unterscheiden, die in keinem dialekt vollständig unterschieden werden, aber in ihrem zusammenwirken die verteilung von oo/au bzw. ee/ei in den unterschiedlichen dialekten erklären. auch bei der morphologie: nordniedersächsisch und mecklenburfisch sind apokopierend, westfälisch nicht, man bekommt also "ursprünglichere" Endungen und silben. das kann helfen, die phonologie zu klären, da "überlänge" oft nicht geschrieben wird. nochmal phonologie: plattmakers bietet IPA, alle anderen nur (mehr oder minder defektive) orthographien. dadurch ist eine disambiguierung der phonologie möglich.

% too linguistic
% Also, the data sources complement each other. We have definitions and attestations from multiple dictionaries. This applies to phonology as well: a comprehensive representation of phonology requires distinguishing a complex inventory of modern reflects of Middle Low German ê and ô phonemes (ê¹, ê², ê³; ô¹, ô², ô³; plus the umlaut forms of ô). In none of the dialects considered here, these are fully differentiated, but only collectively, they account for the distribution of resp. /e:/ $\sim$ /ɛɪ/ $\sim$ /aɪ/, /o:/ $\sim$ /ɔʊ/ $\sim$ /aʊ/, and /ø:/ $\sim$ /œʏ/ $\sim$ /ɔɪ/ in different modern dialects \cite{seelmann1908mundart}. Morphology, too, varies: Northern Low Saxon and Mecklenburgian exhibit apocope, whereas Westphalian does not, preserving ``original'' endings and syllables. If a Westphalian form is linked with a North Low Saxon form, this can aid phonological disambiguation, as apocope and syncope are compensated by lengthening of preceding vowels, but this is often not explicitly marked in writing (and in fact, this is a systematic shortcoming of all German-based orthographies of Low German). In addition, some external dictionaries actually provide phonological forms in IPA (esp., Plattmakers), and this can be exploited to compensate the shortcomings of the orthographic spelling conventions adopted and thus enables phonological disambiguation across dialects.

While the linking method employed here primarily serves to establish a baseline for future research, our cross-dialectal dictionary serves as a testbed for a number of community standards for machine-readable dictionaries on the web in general, and for non-standardized, low-resource languages in particular.
We observed a number of potential gaps in the existing OntoLex vocabularies. 

\begin{enumerate}
\item As our interdialectal links are created by heuristic means, we would like to be able to express to what extend a user can rely on the information conveyed by a link. This includes \emph{candidate links} (with a property such as `\onto{...:possibleMatch}'), but also the possibility to mark a link as an unverified (and eventually, as a verified) hypothesis.
\item More generally, it would be good to have a standard vocabulary for confidence in OntoLex, resp., LexInfo. 
PROV-O \cite{jing2015prov} does not provide a codified vocabulary for expression confidence scores, in fact, the PROV-O documentation has an example that uses a \emph{local} property to provide that information, and PROV-O users have resorted to their own properties, too, e.g., \code{nif:taIdentConf}, \code{nif:taClassConf}, or \code{nif:confidence} in the NLP Interchange Format.\footnote{\url{https://nif.readthedocs.io/en/latest/prov-and-conf.html}} But these properties are designed for a different purpose (linguistic annotation) and should not be applied to lexical linking. It should be noted that confidence scores are a recurring component of lexical resources, but apparently, no standard practice has been established in that regard.
%\footnote{
 %   For example, \cite{declerck2019using} actually provide examples with confidence scores in their source data, but exclude them from the conversion, probably due to a lack of a standard vocabulary.
%} 
More generally, this is an intensely researched problem in the RDF world, and one of the key motivations behind RDF-star \cite{rupp2024implementing}.\footnote{\url{https://www.w3.org/groups/wg/rdf-star/}}
\item Lexinfo currently does not support the reification of \onto{lexinfo:geographicalVariant} (and its sibling properties). As we have to point with \onto{lexinfo:category} to an object property, we move the entire dataset out of the realm of OWL2/DL and into OWL2/Full. As a result, standard reasoning techniques cannot be applied to the resulting lexical knowledge graph. It would be ideal, if there would be an individual with a similar meaning.
\end{enumerate}

In addition to this, we found some solutions for apparent OntoLex gaps, and these may even entail future simplifications:
As such, there is an apparent gap of a counterpart of translation sets for relations other than translations in OntoLex-VarTrans, 
%However, introducing these for any kind of lexical-semantic relation would obfuscate the model. In fact, 
but we found an acceptable work-around in \onto{dct:Dataset}, and we would suggest this as a best practice for other types of lexical-semantic relations, as well. Yet, to align this approach better with the current treatment of translation( set)s, one may consider to  re-define \onto{vartrans:TranslationSet} as a subclass of \onto{dct:Dataset} (and \onto{vartrans:trans} as a subproperty of \onto{dct:hasPart}) and to motivate it as such in a future revision of the VarTrans module. This would be a backward-compatible revision that comes without any additional overhead (i.e. newly introduced concepts). A more radical alternative would be to deprecate \onto{vartrans:TranslationSet} and to refer \onto{dct:Dataset}, instead.

\ign{
    Finally, a remark on language tags for Low German: 
    The current ISO 639 language tags for Low German language varieties do not correspond to linguistically well-defined areas, but they involve a political dimension.
    In particular, all regional dialects of Low German in the Netherlands have their own ISO 639-3 tag in addition to the ISO 639-2 tag \code{nds}, the `standard' tag for Low German is that alludes to the self-designation \word{Nedersaksisch} (in the Netherlands), resp., \word{Nedersassisch} (in Western parts of Germany; another self-designation is \word{Plattdüütsch}, which also is the preferred self-designation in some dialects, including the emmigrant variety of \word{Plautdietsch} that was, accordingly, assigned the ISO 639-3 language tag \code{pdt}). On the other hand, language tags for dialects outside the Netherlands are much less fine-grained, so that the six Dutch Westphalian dialects actually describe subsets of \code{wep} (for Westphalian). As a result, there is some confusion as to which varieties \code{nds} and \code{wep} actually refer to, and we see a lot of diversity in the use of language tags on the web. Even though ISO 639-3 \emph{seems} to restrict \code{nds} to Low German varieties in Germany (because all Dutch varieties of Low German have their own ISO 639-3 language tag), the largest collection of digital data of Low German from the Netherlands is actually designated by the BCP47 code \code{nds-nl}, i.e., `(German?) Low German in the Netherlands'.\footnote{\url{https://nds-nl.wikipedia.org}} 
    With two `major' language tags applied to West Low German (\code{nds} and \code{wep}; plus \code{frs} for Low Saxon East Frisian), it has also become unclear how to properly identify East Low German. Glottolog.org resorted to redefine \code{nds} as `East Low German' and to address all non-Westphalian varieties of West Low German as \code{frs} -- although this is defined as `(Low Saxon) East Frisian'.\footnote{
      The tag \code{frs} was introduced for `East Frisian' in ISO 693-2, but defined there merely by a name list and without extensive documentation. Subsequently, ISO 693-3 assumed that this refers to Low Saxon East Frisian, and thus introduced a novel language tag \code{stq} for East Frisian proper (the Frisian dialect that was largely replaced by Low Saxon East Frisian), which was assumed to be lacking. However, linguistically, Low Saxon East Frisian is a variety of North Low Saxon and probably received this assignment by error, or by analogy with the main varieties of the Frisian (non-Low Saxon) language, \code{fry} (West Frisian) and \code{frr} (North Frisian).
    } In the reality of the web, however, \code{nds} is used for Westphalian and North Low Saxon varieties of the Netherlands (e.g., in the Dutch Low German Wikipedia), as well as North Low Saxon in Germany (in the [German] Low German Wikipedia, written in accordance with Sass, i.e., \code{nds-x-hols1234} for Holstein North Low Saxon, albeit as an orthographic standard, not as a standard variety).\footnote{
        \url{https://nds.wikipedia.org/wiki/Wikipedia:Platt,_wo_schriev_ik_dat\%3F}
    }
    Here, we take \code{nds} to include all parts of North Low Saxon (in Germany) and Mecklenburg-Western Pomeranian, but to exclude Dutch Low German, Westphalian and Low Saxon East Frisian.\footnote{
      We make no claims about the use of language tags for varieties not covered in our sample of dictionaries, in particular, Eastphalian, North and Central Markian, Central Pomeranian, Eastern Pomeranian (extinct) and Low Prussian (extinct). Traditionally, Eastphalian, Westphalian and North Low Saxon (including the Dutch dialects) are grouped together as Western Low German, whereas Mecklenburgian, Pomeranian, Markian and Low Prussian varieties are grouped together as Eastern Low German.
    }
    However, this is problematic insofar as \code{nds} does not refer to a linguistically well-defined group of dialects, and we would like to encourage future discussions on revising the ISO 639-3 language tags, accordingly.
}

Overall, we succeeded in creating our `Rosetta stone' for almost the entirety of Low German in the sense that there now is a human- and machine-readable lexical knowledge graph of (North Low Saxon) lemmas and their interdialectal links into other, externally hosted dictionaries.
However, while we were using standard LLOD technologies to implement this interdialectal linking, we did not actually provide Linguistic Linked Open Data. Our \emph{WöWö} data is linked with dictionaries in HTML, but not RDF -- with a notable exception being Plattmakers, which contains JSON-LD metadata. 
Furthermore, most of these linked data sources (including Plattmakers) are not actually `open' in the sense of the Open Definition. 
But our work represents a first step towards putting Low German on the map of Linguistic Linked Open Data, and a proof-of-principle of its capabilities.
%In fact, it is a pity that LOD technology is not more widely supported by Wörterbuchnetz and DWN, as these are supported by researchers working in the field, and as they are actually designed to provide inter-dictionary links, although only \emph{within their own platform}. 
%However, we hope that we've been able to demonstrate the added value of inter-dictionary links for Low German, and this solution organically extends to any other dictionary available via DWN or Wörterbuchnetz. Also, with technologies such as JSON-LD and RDFa, both these portals can actually be easily extended to provide LOD support on their own -- and to the best we can tell with minimal changes to their current workflows.
A future direction may thus be to encourage or to support the colleagues developing Wörterbuchnetz, DWN, and other platforms, to embrace RDF technologies, and then, to really create an interdialectal, distributed meta-dictionary of Low German, and to facilitate the development of technologies and resources that benefit \emph{all} its varieties in their entirety.

The dataset is planned to be published under a Creative Commons license via \href{https://github.com/nds-spraakverarbeiden/}{nds-spraakverarbeiden on GitHub}. However, we are still in discussions with the \emph{WöWö} author, Peter Neuber about which CC license to take, so that no data has been released yet. This is expected to conclude by end of March, 2025. For reviewing purposes, code and data are temporarily available under \url{https://megastore.rz.uni-augsburg.de/get/4LCORsK95V/}.