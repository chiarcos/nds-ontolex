We describe the creation of a cross-dialectal lexical resource for Low German, a regional language spoken primarily in Germany and the Netherlands, based on the application of Linguistic Linked Open Data (LLOD) technologies. We argue that this approach os particularly well-suited for a language without a written standard, but with multiple, incompatible orthographies and considerable internal variation in phonology, spelling and grammar. A major hurdle in the preservation and documentation of and in the creation of educational materials such as texts and dictionaries for this variety is its internal degree of linguistic and orthographic variation, intensified by mutually exclusive influences from different national languages and their respective orthographies.% in Germany, the Netherlands and in emmigrant communities of speakers
We thus aim to provide a ``digital Rosetta stone'' to unify lexical materials from different dialects through linking dictionaries and mapping corresponding words without the need for a standard variety. This involves two components, a mapping between different orthographies and phonological systems, and a technologies for linking regional dictionaries maintained by different hosts and developed by or for different communities of speakers.
