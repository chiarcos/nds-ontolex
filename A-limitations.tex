\section*{Limitations}

% Since December 2023, a "Limitations" section has been required for all papers submitted to ACL Rolling Review (ARR). This section should be placed at the end of the paper, before the references. The "Limitations" section (along with, optionally, a section for ethical considerations) may be up to one page and will not count toward the final page limit. Note that these files may be used by venues that do not rely on ARR so it is recommended to verify the requirement of a "Limitations" section and other criteria with the venue in question.

The WöWö dictionary considered here is extremely rich and extremely dense in lexiographic and linguistic information, but this is provided in a semistructured format tailored towards human readers. As a result, it is virtually impossible to provide an exhaustive RDF formalization of its content. Instead, we extracted core aspects and demonstrated how these can be extended to include references to other dictionaries that may provide additional information, e.g., examples or (region-specific) definitions.

There are some limitations to our approach of linking. In particular, we excluded interlingual links, resp., additional translations, provided by the indexed dictionaries because of copyright considerations. Neither was part-pof-speech information included. Instead, we provide a purely form-based linking. While this may be further explored (both in its legal dimension and technical applications) in the future, the aim of the current paper is to develop a proof-of-principle implementation for an LLOD-based interdialectal lexical resource for the major dialects of Modern Low German, based on linking existing lexical resources accessible over the web with an RDF conversion of a North Low Saxon dictionary (WöWö). Specific challenges encountered include the complexity of the WöWö dictionary layout, legal constraints regarding the re-usability of external resources linked and aspects of data modelling. Interdialectal linking is solely predicted on grounds of formal agreement, without taking sense information or translations into account.

The primary concern of the paper is technical in nature, with a focus on normalizing divergent orthographies and phonological differences on the one hand, and on modelling issues on the other. We did not systematically assess the linguistic quality of the different dictionaries nor did we evaluate the quality of the generated links. 

In the absence of training data for interlingual mapping, we implement a linguistically informed normalization by means of traditional symbolic methods, and generate candidate matches between normalized source language lemmas and normalized WöWö lemmas, ranked by a confidence score that captures the level of (formal) ambiguity in n:m mappings. While the phonological correspondences are well understood and uncontroversial, this employs solely formal criteria, and is prone to link formally similar but semantically unrelated lemmas. This can be addressed in the future in different ways, e.g., by comparing translations and definitions provided by different dictionaries. This can be implemented by different techniques, including neural methods or lexical linking of existing lexical resources for German, Dutch and English (the primary languages of definitions), but is left as a topic for future research. 
